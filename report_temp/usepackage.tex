\usepackage{luatexja-fontspec}
  \setmainfont{Times New Roman}
  \setsansfont{Arial}
  \setmonofont{PlemolJP}
  \setmainjfont[
    BoldFont            = Noto Sans JP-Bold,
    ItalicFont          = Noto Serif JP,
    ItalicFeatures      = {FakeSlant=0.3},
    BoldItalicFont      = Noto Sans JP-Bold,
    BoldItalicFeatures  = {FakeSlant=0.3},
  ]{Noto Serif JP}
  \setsansjfont[
    ItalicFont          = Noto Sans JP,
    ItalicFeatures      = {FakeSlant=0.3},
    BoldItalicFont      = Noto Sans JP-Bold,
    BoldItalicFeatures  = {FakeSlant=0.3},
  ]{Noto Sans JP}
  \setmonojfont{PlemolJP}
  % 指定フォントが{Times New Roman} {MS Mincho} {MS Gothic} 等の可能性あり

\usepackage{longtable, colortab, colortbl, arydshln}  % 表とその点線
\usepackage{mathtools, breqn, amssymb}            % 数式
  \mathtoolsset{showonlyrefs=true}
\usepackage[margin=25truemm]{geometry}            % 余白
\usepackage{comment, emoji, xurl}                 % コメント,絵文字, URL
\usepackage{titling, tocloft, appendix}           % タイトル,目次,付録
\renewcommand{\cftchapnumwidth}{40pt}             % 目次の被り回避
\renewcommand{\cfttoctitlefont}{\huge\bfseries}   % "目次"の文字サイズ調整                      

\usepackage{graphicx, here}                       % 画像,貼る場所
\usepackage{xcolor}                               % 色
  \definecolor{cornellred}{rgb}{0.7, 0.11, 0.11}
  \definecolor{ceruleanblue}{rgb}{0.16, 0.32, 0.75}
  \definecolor{ferngreen}{rgb}{0.31, 0.47, 0.26}
  \definecolor{whitesmoke}{rgb}{0.97, 0.97, 0.97}

\usepackage{tikz}                                 % 描画
  \usetikzlibrary{intersections, calc, arrows.meta}

\usepackage{ifthen}                               % 条件分岐と反復処理
\usepackage{coffeestains, scsnowman, randomwalk}  % おまけ

\usepackage{pdfpages}                             % PDFの結合

\usepackage{listings}                             % ソースコードの挿入
  \lstset{
    basicstyle          =   \ttfamily,
    keywordstyle        =   \color{cornellred},   
    backgroundcolor     =   \color{whitesmoke},
    identifierstyle     =   \color{ceruleanblue},
    commentstyle        =   \color{ferngreen},
    numberstyle         =   \footnotesize,
    numbers             =   left,
    stepnumber          =   1,
    numbersep           =   15pt,
    frame               =   lines,
    frameround          =   ffff,
    framesep            =   5pt,
    breaklines          =   true,
    breakautoindent     =   true,
    breakatwhitespace   =   true,
    breakindent         =   25pt,
    showspaces          =   false,
    showstringspaces    =   false,
    showtabs            =   true,
    tabsize             =   2,
    captionpos          =   b,
    linewidth           =   \textwidth,
  }
  \renewcommand{\lstlistingname}{ソースコード}

\usepackage[luatex]{hyperref}                     % ハイパーリンク
  \hypersetup{
    setpagesize       = false,
    bookmarksnumbered = true,
    bookmarksopen     = true,
    colorlinks        = true,
    linkcolor         = black,
    citecolor         = black,
    urlcolor          = blue,
  }

\NewPageStyle{myArticle}{
  nombre_position       = top-right,
  nombre                = \arabic{page},
  running_head_position = top-left,
  odd_running_head      = _section,
  }

\NewPageStyle{myReport}{
  nombre                = \roman{page},
  nombre_position       = top-right,
}

% 目次ページでページスタイルを無効化しないための呪文
\AtBeginDocument{\addtocontents{toc}{\protect\thispagestyle{myReport}}}

% ----------hoge----------

\usepackage[
  style       = numeric-comp,
  sorting     = none,
  sortcites   = true,
  url         = false,
  isbn        = false,
  backref     = false,
  date        = year,
  giveninits  = true,
]{biblatex}

% DOIの頭のhttp:// を除去
\DeclareSourcemap{
  \maps[datatype=bibtex]{
    \map{
      \step[
      fieldsource = doi,
      match       = \regexp{https?://(dx.)?doi.org/(.+)},
      replace     = \regexp{$2}
      ]
    }
  }
}

% eprintがあるときはDOIを消去
\AtEveryBibitem{\iffieldundef{eprint}{}{\clearfield{doi}}}

% 完全新規のマクロ
\newbibmacro*{bbx:parunit}{%
  \ifbibliography
    {\setunit{\bibpagerefpunct}\newblock
     \usebibmacro{pageref}%
     \clearlist{pageref}%
     \setunit{\adddot\\\nobreak}%改行
    }
  {}%
}

% 既存のマクロを再定義,リンクの前で改行
\renewbibmacro*{doi+eprint+url}{%
  \usebibmacro{bbx:parunit}% Added
  \iftoggle{bbx:doi}{\printfield{doi}}{}%
  \iftoggle{bbx:eprint}{\usebibmacro{eprint}}{}%
}

\renewbibmacro*{eprint}{%
  \usebibmacro{bbx:parunit}% Added
  \iffieldundef{eprinttype}
    {\printfield{eprint}}
    {\printfield[eprint:\strfield{eprinttype}]{eprint}}}

\addbibresource{bunken.bib}

% ----------hoge----------
