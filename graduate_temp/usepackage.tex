\usepackage{luatexja-fontspec}
  \setmainfont{Century Schoolbook}
  \setsansfont{Arial}
  \setmonofont{PlemolJP}
  \setmainjfont[
    BoldFont            = Noto Sans JP-Bold,
    ItalicFont          = Noto Serif JP,
    ItalicFeatures      = {FakeSlant=0.3},
    BoldItalicFont      = Noto Sans JP-Bold,
    BoldItalicFeatures  = {FakeSlant=0.3},
  ]{Noto Serif JP}
  \setsansjfont[
    ItalicFont          = Noto Sans JP,
    ItalicFeatures      = {FakeSlant=0.3},
    BoldItalicFont      = Noto Sans JP-Bold,
    BoldItalicFeatures  = {FakeSlant=0.3},
  ]{Noto Sans JP}
  \setmonojfont{PlemolJP}
  % 指定フォントが{Times New Roman} {MS Mincho} {MS Gothic} 等の可能性あり

\usepackage{mathtools, autobreak, amssymb}      % 数式
\usepackage[margin=25truemm]{geometry}          % 余白
\usepackage{comment, emoji, xurl}               % コメント,絵文字, URL
\usepackage{titling, tocloft, appendix}         % タイトル,目次,付録
\renewcommand{\cftchapnumwidth}{40pt}             % 目次の被り回避
\renewcommand{\cfttoctitlefont}{\huge\bfseries}   % "目次"の文字サイズ調整
                        

\usepackage{graphicx, here}                       % 画像,貼る場所
\usepackage[dvipsnames]{xcolor}                   % 色
\usepackage{tikz}                                 % 描画
  \usetikzlibrary{intersections, calc, arrows.meta}

\usepackage{ifthen}                               % 条件分岐と反復処理
\usepackage{coffeestains, scsnowman, randomwalk}  % おまけ

\usepackage{listings}                             % ソースコードの挿入
  \lstset{
    basicstyle          =   \ttfamily,
    keywordstyle        =   \color{blue},
    stringstyle         =   \color{RedOrange},
    commentstyle        =   \color{CadetBlue},
    rulecolor           =   \color{darkgray},     
    backgroundcolor     =   \color[RGB]{250,250,250},
    numberstyle         =   \footnotesize,
    numbers             =   left,
    stepnumber          =   1,
    numbersep           =   15pt,
    frame               =   lines,
    frameround          =   ffff,
    framesep            =   5pt,
    breaklines          =   true,
    breakautoindent     =   true,
    breakatwhitespace   =   true,
    breakindent         =   25pt,
    showspaces          =   false,
    showstringspaces    =   false,
    showtabs            =   true,
    tabsize             =   2,
    captionpos          =   b,
    linewidth           =   \textwidth,
  }
  \renewcommand{\lstlistingname}{ソースコード}

\usepackage[luatex]{hyperref}                     % ハイパーリンク
  \hypersetup{
    setpagesize       = false,
    bookmarksnumbered = true,
    bookmarksopen     = true,
    colorlinks        = true,
    linkcolor         = black,
    citecolor         = Blue,
    urlcolor          = Blue,
  }

\NewPageStyle{myArticle}{
  nombre_position       = top-right,
  nombre                = \arabic{page},
  running_head_position = top-left,
  odd_running_head      = _section,
  }

\NewPageStyle{myReport}{
  nombre                = \roman{page},
  nombre_position       = top-right,
}

% ----------hoge----------

\usepackage[
  style   = numeric-verb,
  sorting = none,
  url     = false,
  isbn    = false,
]{biblatex}

% DOIの頭のhttp:// を除去
\DeclareSourcemap{
  \maps[datatype=bibtex]{
    \map{
      \step[
      fieldsource = doi,
      match       = \regexp{https?://(dx.)?doi.org/(.+)},
      replace     = \regexp{$2}
      ]
    }
  }
}

% eprintがあるときはDOIを消去
\AtEveryBibitem{\iffieldundef{eprint}{}{\clearfield{doi}}}

% 完全新規のマクロ
\newbibmacro*{bbx:parunit}{%
  \ifbibliography
    {\setunit{\bibpagerefpunct}\newblock
     \usebibmacro{pageref}%
     \clearlist{pageref}%
     \setunit{\adddot\\\nobreak}%改行
    }
  {}%
}

% 既存のマクロを再定義,リンクの前で改行
\renewbibmacro*{doi+eprint+url}{%
  \usebibmacro{bbx:parunit}% Added
  \iftoggle{bbx:doi}{\printfield{doi}}{}%
  \iftoggle{bbx:eprint}{\usebibmacro{eprint}}{}%
}

\renewbibmacro*{eprint}{%
  \usebibmacro{bbx:parunit}% Added
  \iffieldundef{eprinttype}
    {\printfield{eprint}}
    {\printfield[eprint:\strfield{eprinttype}]{eprint}}}

\addbibresource{bunken.bib}

% ----------hoge----------